% Options for packages loaded elsewhere
\PassOptionsToPackage{unicode}{hyperref}
\PassOptionsToPackage{hyphens}{url}
%
\documentclass[
]{article}
\usepackage{amsmath,amssymb}
\usepackage{iftex}
\ifPDFTeX
  \usepackage[T1]{fontenc}
  \usepackage[utf8]{inputenc}
  \usepackage{textcomp} % provide euro and other symbols
\else % if luatex or xetex
  \usepackage{unicode-math} % this also loads fontspec
  \defaultfontfeatures{Scale=MatchLowercase}
  \defaultfontfeatures[\rmfamily]{Ligatures=TeX,Scale=1}
\fi
\usepackage{lmodern}
\ifPDFTeX\else
  % xetex/luatex font selection
\fi
% Use upquote if available, for straight quotes in verbatim environments
\IfFileExists{upquote.sty}{\usepackage{upquote}}{}
\IfFileExists{microtype.sty}{% use microtype if available
  \usepackage[]{microtype}
  \UseMicrotypeSet[protrusion]{basicmath} % disable protrusion for tt fonts
}{}
\makeatletter
\@ifundefined{KOMAClassName}{% if non-KOMA class
  \IfFileExists{parskip.sty}{%
    \usepackage{parskip}
  }{% else
    \setlength{\parindent}{0pt}
    \setlength{\parskip}{6pt plus 2pt minus 1pt}}
}{% if KOMA class
  \KOMAoptions{parskip=half}}
\makeatother
\usepackage{xcolor}
\usepackage[margin=1in]{geometry}
\usepackage{color}
\usepackage{fancyvrb}
\newcommand{\VerbBar}{|}
\newcommand{\VERB}{\Verb[commandchars=\\\{\}]}
\DefineVerbatimEnvironment{Highlighting}{Verbatim}{commandchars=\\\{\}}
% Add ',fontsize=\small' for more characters per line
\usepackage{framed}
\definecolor{shadecolor}{RGB}{248,248,248}
\newenvironment{Shaded}{\begin{snugshade}}{\end{snugshade}}
\newcommand{\AlertTok}[1]{\textcolor[rgb]{0.94,0.16,0.16}{#1}}
\newcommand{\AnnotationTok}[1]{\textcolor[rgb]{0.56,0.35,0.01}{\textbf{\textit{#1}}}}
\newcommand{\AttributeTok}[1]{\textcolor[rgb]{0.13,0.29,0.53}{#1}}
\newcommand{\BaseNTok}[1]{\textcolor[rgb]{0.00,0.00,0.81}{#1}}
\newcommand{\BuiltInTok}[1]{#1}
\newcommand{\CharTok}[1]{\textcolor[rgb]{0.31,0.60,0.02}{#1}}
\newcommand{\CommentTok}[1]{\textcolor[rgb]{0.56,0.35,0.01}{\textit{#1}}}
\newcommand{\CommentVarTok}[1]{\textcolor[rgb]{0.56,0.35,0.01}{\textbf{\textit{#1}}}}
\newcommand{\ConstantTok}[1]{\textcolor[rgb]{0.56,0.35,0.01}{#1}}
\newcommand{\ControlFlowTok}[1]{\textcolor[rgb]{0.13,0.29,0.53}{\textbf{#1}}}
\newcommand{\DataTypeTok}[1]{\textcolor[rgb]{0.13,0.29,0.53}{#1}}
\newcommand{\DecValTok}[1]{\textcolor[rgb]{0.00,0.00,0.81}{#1}}
\newcommand{\DocumentationTok}[1]{\textcolor[rgb]{0.56,0.35,0.01}{\textbf{\textit{#1}}}}
\newcommand{\ErrorTok}[1]{\textcolor[rgb]{0.64,0.00,0.00}{\textbf{#1}}}
\newcommand{\ExtensionTok}[1]{#1}
\newcommand{\FloatTok}[1]{\textcolor[rgb]{0.00,0.00,0.81}{#1}}
\newcommand{\FunctionTok}[1]{\textcolor[rgb]{0.13,0.29,0.53}{\textbf{#1}}}
\newcommand{\ImportTok}[1]{#1}
\newcommand{\InformationTok}[1]{\textcolor[rgb]{0.56,0.35,0.01}{\textbf{\textit{#1}}}}
\newcommand{\KeywordTok}[1]{\textcolor[rgb]{0.13,0.29,0.53}{\textbf{#1}}}
\newcommand{\NormalTok}[1]{#1}
\newcommand{\OperatorTok}[1]{\textcolor[rgb]{0.81,0.36,0.00}{\textbf{#1}}}
\newcommand{\OtherTok}[1]{\textcolor[rgb]{0.56,0.35,0.01}{#1}}
\newcommand{\PreprocessorTok}[1]{\textcolor[rgb]{0.56,0.35,0.01}{\textit{#1}}}
\newcommand{\RegionMarkerTok}[1]{#1}
\newcommand{\SpecialCharTok}[1]{\textcolor[rgb]{0.81,0.36,0.00}{\textbf{#1}}}
\newcommand{\SpecialStringTok}[1]{\textcolor[rgb]{0.31,0.60,0.02}{#1}}
\newcommand{\StringTok}[1]{\textcolor[rgb]{0.31,0.60,0.02}{#1}}
\newcommand{\VariableTok}[1]{\textcolor[rgb]{0.00,0.00,0.00}{#1}}
\newcommand{\VerbatimStringTok}[1]{\textcolor[rgb]{0.31,0.60,0.02}{#1}}
\newcommand{\WarningTok}[1]{\textcolor[rgb]{0.56,0.35,0.01}{\textbf{\textit{#1}}}}
\usepackage{graphicx}
\makeatletter
\newsavebox\pandoc@box
\newcommand*\pandocbounded[1]{% scales image to fit in text height/width
  \sbox\pandoc@box{#1}%
  \Gscale@div\@tempa{\textheight}{\dimexpr\ht\pandoc@box+\dp\pandoc@box\relax}%
  \Gscale@div\@tempb{\linewidth}{\wd\pandoc@box}%
  \ifdim\@tempb\p@<\@tempa\p@\let\@tempa\@tempb\fi% select the smaller of both
  \ifdim\@tempa\p@<\p@\scalebox{\@tempa}{\usebox\pandoc@box}%
  \else\usebox{\pandoc@box}%
  \fi%
}
% Set default figure placement to htbp
\def\fps@figure{htbp}
\makeatother
\setlength{\emergencystretch}{3em} % prevent overfull lines
\providecommand{\tightlist}{%
  \setlength{\itemsep}{0pt}\setlength{\parskip}{0pt}}
\setcounter{secnumdepth}{-\maxdimen} % remove section numbering
\usepackage{bookmark}
\IfFileExists{xurl.sty}{\usepackage{xurl}}{} % add URL line breaks if available
\urlstyle{same}
\hypersetup{
  pdftitle={group\_act1},
  pdfauthor={Jessamine Paula Orada},
  hidelinks,
  pdfcreator={LaTeX via pandoc}}

\title{group\_act1}
\author{Jessamine Paula Orada}
\date{2025-12-01}

\begin{document}
\maketitle

\begin{Shaded}
\begin{Highlighting}[]
\DocumentationTok{\#\#\# Loading the library}

\FunctionTok{library}\NormalTok{(rvest)}
\FunctionTok{library}\NormalTok{(dplyr)}
\end{Highlighting}
\end{Shaded}

\begin{verbatim}
## 
## Attaching package: 'dplyr'
\end{verbatim}

\begin{verbatim}
## The following objects are masked from 'package:stats':
## 
##     filter, lag
\end{verbatim}

\begin{verbatim}
## The following objects are masked from 'package:base':
## 
##     intersect, setdiff, setequal, union
\end{verbatim}

\begin{Shaded}
\begin{Highlighting}[]
\FunctionTok{library}\NormalTok{(stringr)   }
\FunctionTok{library}\NormalTok{(lubridate) }\DocumentationTok{\#\# For date formats}
\end{Highlighting}
\end{Shaded}

\begin{verbatim}
## 
## Attaching package: 'lubridate'
\end{verbatim}

\begin{verbatim}
## The following objects are masked from 'package:base':
## 
##     date, intersect, setdiff, union
\end{verbatim}

\begin{Shaded}
\begin{Highlighting}[]
\FunctionTok{library}\NormalTok{(ggplot2)}
\end{Highlighting}
\end{Shaded}

\begin{Shaded}
\begin{Highlighting}[]
\DocumentationTok{\#\#\# 1. Creating an object}

\NormalTok{titles }\OtherTok{\textless{}{-}} \FunctionTok{character}\NormalTok{(}\DecValTok{0}\NormalTok{)}
\NormalTok{authors }\OtherTok{\textless{}{-}} \FunctionTok{character}\NormalTok{(}\DecValTok{0}\NormalTok{)}
\NormalTok{submission\_dates }\OtherTok{\textless{}{-}} \FunctionTok{character}\NormalTok{(}\DecValTok{0}\NormalTok{)}
\NormalTok{originally\_announced }\OtherTok{\textless{}{-}} \FunctionTok{character}\NormalTok{(}\DecValTok{0}\NormalTok{)}
\NormalTok{doi }\OtherTok{\textless{}{-}} \FunctionTok{character}\NormalTok{(}\DecValTok{0}\NormalTok{)}
\end{Highlighting}
\end{Shaded}

\begin{Shaded}
\begin{Highlighting}[]
\DocumentationTok{\#\#\# 2. Importing the url and created a structure}

\CommentTok{\# Base URL for Solar and Stellar Astrophysics (astro{-}ph.SR)}
\NormalTok{base\_url }\OtherTok{\textless{}{-}} \StringTok{"https://arxiv.org/search/?query=astro{-}ph.SR\&searchtype=all\&source=header\&start="}
\end{Highlighting}
\end{Shaded}

\begin{Shaded}
\begin{Highlighting}[]
\NormalTok{all\_papers }\OtherTok{\textless{}{-}} \FunctionTok{list}\NormalTok{()}



\CommentTok{\# Loop 4 times to get 200 papers (0, 50, 100, 150)}
\NormalTok{starts }\OtherTok{\textless{}{-}} \FunctionTok{seq}\NormalTok{(}\AttributeTok{from =} \DecValTok{0}\NormalTok{, }\AttributeTok{to =} \DecValTok{150}\NormalTok{, }\AttributeTok{by =} \DecValTok{50}\NormalTok{)}

\ControlFlowTok{for}\NormalTok{ (i }\ControlFlowTok{in}\NormalTok{ starts) \{}
  
  \CommentTok{\# Construct URL}
\NormalTok{  url }\OtherTok{\textless{}{-}} \FunctionTok{paste0}\NormalTok{(base\_url, i)}
  \FunctionTok{print}\NormalTok{(}\FunctionTok{paste}\NormalTok{(}\StringTok{"Scraping:"}\NormalTok{, url)) }\CommentTok{\# Print progress so you know it\textquotesingle{}s working}
  
  \CommentTok{\# STANDARD SCRAPING (Replaces polite::scrape)}
  \CommentTok{\# We use tryCatch to skip a page if an error occurs, preventing a total crash}
  \FunctionTok{tryCatch}\NormalTok{(\{}
\NormalTok{    page }\OtherTok{\textless{}{-}} \FunctionTok{read\_html}\NormalTok{(url)}
    
    \CommentTok{\# Extract containers}
\NormalTok{    papers\_html }\OtherTok{\textless{}{-}}\NormalTok{ page }\SpecialCharTok{\%\textgreater{}\%} \FunctionTok{html\_nodes}\NormalTok{(}\StringTok{"li.arxiv{-}result"}\NormalTok{)}
    
    \CommentTok{\# Extract Data Elements}
\NormalTok{    titles }\OtherTok{\textless{}{-}}\NormalTok{ papers\_html }\SpecialCharTok{\%\textgreater{}\%} 
      \FunctionTok{html\_node}\NormalTok{(}\StringTok{"p.title.is{-}5.mathjax"}\NormalTok{) }\SpecialCharTok{\%\textgreater{}\%} 
      \FunctionTok{html\_text}\NormalTok{(}\AttributeTok{trim =} \ConstantTok{TRUE}\NormalTok{)}
    
\NormalTok{    authors }\OtherTok{\textless{}{-}}\NormalTok{ papers\_html }\SpecialCharTok{\%\textgreater{}\%} 
      \FunctionTok{html\_node}\NormalTok{(}\StringTok{"p.authors"}\NormalTok{) }\SpecialCharTok{\%\textgreater{}\%} 
      \FunctionTok{html\_text}\NormalTok{(}\AttributeTok{trim =} \ConstantTok{TRUE}\NormalTok{) }\SpecialCharTok{\%\textgreater{}\%} 
      \FunctionTok{str\_remove}\NormalTok{(}\StringTok{"Authors:}\SpecialCharTok{\textbackslash{}n}\StringTok{"}\NormalTok{)}
    
\NormalTok{    abstracts }\OtherTok{\textless{}{-}}\NormalTok{ papers\_html }\SpecialCharTok{\%\textgreater{}\%} 
      \FunctionTok{html\_node}\NormalTok{(}\StringTok{"span.abstract{-}full"}\NormalTok{) }\SpecialCharTok{\%\textgreater{}\%} 
      \FunctionTok{html\_text}\NormalTok{(}\AttributeTok{trim =} \ConstantTok{TRUE}\NormalTok{) }\SpecialCharTok{\%\textgreater{}\%} 
      \FunctionTok{str\_remove}\NormalTok{(}\StringTok{"▽ Less"}\NormalTok{)}
    
\NormalTok{    meta\_raw }\OtherTok{\textless{}{-}}\NormalTok{ papers\_html }\SpecialCharTok{\%\textgreater{}\%} 
      \FunctionTok{html\_node}\NormalTok{(}\StringTok{"p.is{-}size{-}7"}\NormalTok{) }\SpecialCharTok{\%\textgreater{}\%} 
      \FunctionTok{html\_text}\NormalTok{(}\AttributeTok{trim =} \ConstantTok{TRUE}\NormalTok{)}
    
    \CommentTok{\# Store in temporary dataframe}
\NormalTok{    temp\_df }\OtherTok{\textless{}{-}} \FunctionTok{data.frame}\NormalTok{(}
      \AttributeTok{title =}\NormalTok{ titles,}
      \AttributeTok{author =}\NormalTok{ authors,}
      \AttributeTok{abstract =}\NormalTok{ abstracts,}
      \AttributeTok{meta\_raw =}\NormalTok{ meta\_raw,}
      \AttributeTok{stringsAsFactors =} \ConstantTok{FALSE}
\NormalTok{    )}
    
\NormalTok{    all\_papers[[}\FunctionTok{length}\NormalTok{(all\_papers) }\SpecialCharTok{+} \DecValTok{1}\NormalTok{]] }\OtherTok{\textless{}{-}}\NormalTok{ temp\_df}
    
\NormalTok{  \}, }\AttributeTok{error =} \ControlFlowTok{function}\NormalTok{(e) \{}
    \FunctionTok{print}\NormalTok{(}\FunctionTok{paste}\NormalTok{(}\StringTok{"Error on page starting at"}\NormalTok{, i))}
\NormalTok{  \})}
  
  \CommentTok{\# IMPORTANT: Wait 3 seconds between pages to avoid being banned by arXiv}
  \FunctionTok{Sys.sleep}\NormalTok{(}\DecValTok{3}\NormalTok{) }
\NormalTok{\}}
\end{Highlighting}
\end{Shaded}

\begin{verbatim}
## [1] "Scraping: https://arxiv.org/search/?query=astro-ph.SR&searchtype=all&source=header&start=0"
## [1] "Scraping: https://arxiv.org/search/?query=astro-ph.SR&searchtype=all&source=header&start=50"
## [1] "Scraping: https://arxiv.org/search/?query=astro-ph.SR&searchtype=all&source=header&start=100"
## [1] "Scraping: https://arxiv.org/search/?query=astro-ph.SR&searchtype=all&source=header&start=150"
\end{verbatim}

\begin{Shaded}
\begin{Highlighting}[]
\CommentTok{\# Combine all lists into one dataframe}
\NormalTok{df\_papers }\OtherTok{\textless{}{-}} \FunctionTok{bind\_rows}\NormalTok{(all\_papers)}

\CommentTok{\# Check count}
\FunctionTok{print}\NormalTok{(}\FunctionTok{paste}\NormalTok{(}\StringTok{"Total papers extracted:"}\NormalTok{, }\FunctionTok{nrow}\NormalTok{(df\_papers)))}
\end{Highlighting}
\end{Shaded}

\begin{verbatim}
## [1] "Total papers extracted: 200"
\end{verbatim}

\subsubsection{3. Cleaning the data}\label{cleaning-the-data}

\begin{Shaded}
\begin{Highlighting}[]
\NormalTok{df\_clean }\OtherTok{\textless{}{-}}\NormalTok{ df\_papers }\SpecialCharTok{\%\textgreater{}\%} 
  \FunctionTok{mutate}\NormalTok{(}
    \CommentTok{\# 1. Extract Submission Date}
    \AttributeTok{submission\_date\_text =} \FunctionTok{str\_extract}\NormalTok{(meta\_raw, }\StringTok{"Submitted.*?(=?;)"}\NormalTok{),}
    \AttributeTok{submission\_date\_text =} \FunctionTok{str\_remove\_all}\NormalTok{(submission\_date\_text, }\StringTok{"Submitted |;"}\NormalTok{),}
    \AttributeTok{submission\_date =} \FunctionTok{dmy}\NormalTok{(submission\_date\_text),}
    
    \CommentTok{\# 2. Extract DOI}
    \AttributeTok{doi =} \FunctionTok{str\_extract}\NormalTok{(meta\_raw, }\StringTok{"doi:.*"}\NormalTok{),}
    \AttributeTok{doi =} \FunctionTok{str\_remove}\NormalTok{(doi, }\StringTok{"doi:"}\NormalTok{),}
    
    \CommentTok{\# 3. Extract Announced Date (Backup if submission is missing)}
    \AttributeTok{announced\_date\_text =} \FunctionTok{str\_extract}\NormalTok{(meta\_raw, }\StringTok{"originally announced [A{-}Za{-}z]+ [0{-}9]\{4\}"}\NormalTok{),}
    \AttributeTok{announced\_date\_text =} \FunctionTok{str\_remove}\NormalTok{(announced\_date\_text, }\StringTok{"originally announced "}\NormalTok{),}
    \AttributeTok{originally\_announced =} \FunctionTok{my}\NormalTok{(announced\_date\_text)}
\NormalTok{  )}
\end{Highlighting}
\end{Shaded}

\section{Remove rows where date might have failed
(optional)}\label{remove-rows-where-date-might-have-failed-optional}

\begin{Shaded}
\begin{Highlighting}[]
\NormalTok{df\_clean }\OtherTok{\textless{}{-}}\NormalTok{ df\_clean }\SpecialCharTok{\%\textgreater{}\%} \FunctionTok{filter}\NormalTok{(}\SpecialCharTok{!}\FunctionTok{is.na}\NormalTok{(submission\_date))}

\FunctionTok{head}\NormalTok{(df\_clean }\SpecialCharTok{\%\textgreater{}\%} \FunctionTok{select}\NormalTok{(title, submission\_date, doi))}
\end{Highlighting}
\end{Shaded}

\begin{verbatim}
##                                                                                             title
## 1                      The EBLM Project XVIII. 3D Obliquities of Five Low-Mass Eclipsing Binaries
## 2                                The Kinematic Properties of TŻO Candidate HV 11417 with Gaia DR3
## 3                                         CO observations in rotating circumbinary post-AGB disks
## 4                         Properties of Core Collapse Supernovae from Binary Population Synthesis
## 5 Predictions on gravity mode pulsations of collisional blue straggler stars in globular clusters
## 6           Large-amplitude Variability Driven by Giant Dust Storms on a Planetary-mass Companion
##   submission_date  doi
## 1      2025-11-28 <NA>
## 2      2025-11-28 <NA>
## 3      2025-11-28 <NA>
## 4      2025-11-28 <NA>
## 5      2025-11-28 <NA>
## 6      2025-11-28 <NA>
\end{verbatim}

\subsubsection{4. Arranging the dates}\label{arranging-the-dates}

\begin{Shaded}
\begin{Highlighting}[]
\NormalTok{df\_sorted }\OtherTok{\textless{}{-}}\NormalTok{ df\_clean }\SpecialCharTok{\%\textgreater{}\%} 
  \FunctionTok{arrange}\NormalTok{(submission\_date)}
\end{Highlighting}
\end{Shaded}

\section{Display summary statistics}\label{display-summary-statistics}

\begin{Shaded}
\begin{Highlighting}[]
\FunctionTok{print}\NormalTok{(}\FunctionTok{paste}\NormalTok{(}\StringTok{"Date range:"}\NormalTok{, }\FunctionTok{min}\NormalTok{(df\_sorted}\SpecialCharTok{$}\NormalTok{submission\_date), }\StringTok{"to"}\NormalTok{, }\FunctionTok{max}\NormalTok{(df\_sorted}\SpecialCharTok{$}\NormalTok{submission\_date)))}
\end{Highlighting}
\end{Shaded}

\begin{verbatim}
## [1] "Date range: 2025-11-14 to 2025-11-28"
\end{verbatim}

\begin{Shaded}
\begin{Highlighting}[]
\FunctionTok{print}\NormalTok{(}\FunctionTok{paste}\NormalTok{(}\StringTok{"Total papers after cleaning:"}\NormalTok{, }\FunctionTok{nrow}\NormalTok{(df\_sorted)))}
\end{Highlighting}
\end{Shaded}

\begin{verbatim}
## [1] "Total papers after cleaning: 200"
\end{verbatim}

\subsubsection{5. Turning into a plot time
series}\label{turning-into-a-plot-time-series}

\begin{Shaded}
\begin{Highlighting}[]
\CommentTok{\# Count papers per month}
\NormalTok{papers\_per\_month }\OtherTok{\textless{}{-}}\NormalTok{ df\_sorted }\SpecialCharTok{\%\textgreater{}\%}
  \FunctionTok{mutate}\NormalTok{(}\AttributeTok{month\_year =} \FunctionTok{floor\_date}\NormalTok{(submission\_date, }\StringTok{"month"}\NormalTok{)) }\SpecialCharTok{\%\textgreater{}\%}
  \FunctionTok{group\_by}\NormalTok{(month\_year) }\SpecialCharTok{\%\textgreater{}\%}
  \FunctionTok{summarise}\NormalTok{(}\AttributeTok{count =} \FunctionTok{n}\NormalTok{())}

\CommentTok{\# Plot}
\FunctionTok{ggplot}\NormalTok{(papers\_per\_month, }\FunctionTok{aes}\NormalTok{(}\AttributeTok{x =}\NormalTok{ month\_year, }\AttributeTok{y =}\NormalTok{ count)) }\SpecialCharTok{+}
  \FunctionTok{geom\_line}\NormalTok{(}\AttributeTok{color =} \StringTok{"darkblue"}\NormalTok{, }\AttributeTok{size =} \DecValTok{1}\NormalTok{) }\SpecialCharTok{+}
  \FunctionTok{geom\_point}\NormalTok{(}\AttributeTok{color =} \StringTok{"red"}\NormalTok{) }\SpecialCharTok{+}
  \FunctionTok{labs}\NormalTok{(}\AttributeTok{title =} \StringTok{"Time Series: Solar and Stellar Astrophysics Papers (arXiv)"}\NormalTok{,}
       \AttributeTok{subtitle =} \StringTok{"Frequency of papers submitted per month (astro{-}ph.SR)"}\NormalTok{,}
       \AttributeTok{x =} \StringTok{"Date"}\NormalTok{,}
       \AttributeTok{y =} \StringTok{"Number of Papers"}\NormalTok{) }\SpecialCharTok{+}
  \FunctionTok{theme\_minimal}\NormalTok{()}
\end{Highlighting}
\end{Shaded}

\begin{verbatim}
## Warning: Using `size` aesthetic for lines was deprecated in ggplot2 3.4.0.
## i Please use `linewidth` instead.
## This warning is displayed once every 8 hours.
## Call `lifecycle::last_lifecycle_warnings()` to see where this warning was
## generated.
\end{verbatim}

\begin{verbatim}
## `geom_line()`: Each group consists of only one observation.
## i Do you need to adjust the group aesthetic?
\end{verbatim}

\pandocbounded{\includegraphics[keepaspectratio]{groupact1dec1_files/figure-latex/unnamed-chunk-9-1.pdf}}

\subsubsection{6. Export to CSV
(Optional)}\label{export-to-csv-optional}

\begin{Shaded}
\begin{Highlighting}[]
\CommentTok{\# Export the cleaned data}
\FunctionTok{write.csv}\NormalTok{(df\_sorted, }\StringTok{"arxiv\_astro{-}ph{-}SR\_papers.csv"}\NormalTok{, }\AttributeTok{row.names =} \ConstantTok{FALSE}\NormalTok{)}
\FunctionTok{print}\NormalTok{(}\StringTok{"Data exported to arxiv\_astro{-}ph{-}SR\_papers.csv"}\NormalTok{)}
\end{Highlighting}
\end{Shaded}

\begin{verbatim}
## [1] "Data exported to arxiv_astro-ph-SR_papers.csv"
\end{verbatim}

\end{document}
